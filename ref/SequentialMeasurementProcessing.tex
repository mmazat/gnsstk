\documentclass[letterpaper,11pt]{article}

\setlength{\parindent}{0.0cm}
\setlength{\parskip}{0.2cm}

%opening
\title{Sequential Measurement Processing}
\author{Glenn D. MacGougan}

\begin{document}

\maketitle

\begin{abstract}
In Kalman filtering, it is very useful to be able to process measurements sequentially (i.e. one at a time). This method eliminates the need for matrix inversion operations since sequential processing uses scalar measurement updates $^{[1]}$. The method requires that measurements are uncorrelated. When this is not the case, diagonalizeation of the measurement variance-covariance matrix and a transformation that is a linear combination of the original measurements into a new measurement set is performed. This whitepaper outlines the steps for a sequential measurement processing technique .
\end{abstract}

\section{Best References}

The book by Brown and Hwang $^{[1]}$ discusses sequential measurement processing on pages 250-252.

The book by Grewel and Andrews $^{[2]}$ discusses sequential measurement processing on page 226 in the section entitled ``Kalman Implementation with Decorrelation.''

\pagebreak

\section{The Proposed Method}

The standard Kalman filter steps are:

\begin{enumerate}
  \item Prediction: \\
         $ \hat{x}_{k+1}^- = \phi_k \hat{x}_{k} $ \\
         $ P_{k+1}^- = \phi_k P_k \phi_k^T + Q_k $
  \item Computation of Kalman Gain: \\
         $ K_k = P_k^- H_k^T ( H_k P_k^- H_k^T + R_k ) ^{-1} $
  \item Update: \\
         $ \hat{x}_k^+ = \hat{x}_k^- + K_k(z_k - H_k \hat{x}_k^- ) $\\
         $ P_k = (I-K_k H_k) P_k^- $
\end{enumerate}

The standard Extended Kalman filter steps are:

\begin{enumerate}
  \item Prediction: \\
         $ \hat{x}_{k+1}^- = \phi_k \hat{x}_{k} $ \\
         $ P_{k+1}^- = \phi_k P_k \phi_k^T + Q_k $
  \item Compute Innovations: \\
         $ w = l-\hat{l} $, (measurements - computed (i.e. predicted) measurements). \\
         $ \hat{l} = H_k \hat{x}_{k+1}^- $
  \item Computation of Kalman Gain: \\
         $ K_k = P_k^- H_k^T ( H_k P_k^- H_k^T + R_k ) ^{-1} $
  \item Update: \\
         $ \widehat{dx}_k = K_k(w_k) $\\
         $ \hat{x}_k^+ = \hat{x}_k^- + \widehat{dx}_k $ \\
         $ P_k = (I-K_k H_k) P_k^- $
\end{enumerate}

There is an alternative Kalman filter implementation that propagates $P_k^{-1}$ instead of $P_k$. Since the state variance-covariance information is required for output purposes at each epoch in many cases for GNSS processing, this will not be used.

\pagebreak

This method will use the standard Extended Kalman implementation. The steps are as follows:

\begin{enumerate}
  \item Perform Prediction: \\
         $ \hat{x}_{k+1}^- = \phi_k \hat{x}_{k} $ \\
         $ P_{k+1}^- = \phi_k P_k \phi_k^T + Q_k $
  \item Compute Innovations: \\
         $ w = l-\hat{l} $, (measurements - computed (i.e. predicted) measurements). \\
         $ \hat{l} = H_k \hat{x}_{k+1}^- $
  \item Perform suitable innovation testing to remove outliers.
  \item For each measurement, i, the following steps are performed.
  \begin{enumerate}
    \item Compute the $i^th$ innovation, $w_i$ using the most recent state estimate.
    \item Compute the Kalman gain for the $i^{th}$ measurement:
    \begin{enumerate}
      \item Compute $h_i$, [1xu] the row of the design matrix corresponding to the observation.
      \item Compute $pht = P_k h_i^T$, [ux1].
      \item Compute $C = (h_i P_k h_i^T + \sigma_i^2)^{-1} = (h_i pht + \sigma_i^2)^{-1}$, [1x1]  Hence, the uxu inversion is avoided as only a scalar is inverted. $C$ is the variance of the innovation.
      \item Compute $k_i = P_k h_i^T ( h_i P_k h_i^T + \sigma_i^2)^{-1} = pht C$, [ux1].
    \end{enumerate}
    \item Then the state variance-covariance matrix is updated:
    \begin{enumerate}
      \item Compute $D = k_i h_i P_k$.
      \item Compute $P_k = P_k - D$ (equivalently $P_k = (I-k_i h_i)P_k$.
    \end{enumerate}
    \item Then the states are updated:
    \begin{enumerate}
      \item $ \widehat{dx}_k = k_i(w_i) $
      \item $ \hat{x}_k^+ = \hat{x}_k^- + \widehat{dx}_k $
    \end{enumerate}
  \end{enumerate}
\end{enumerate}



\section{References}
\begin{enumerate}
\item Brown, R. G., P. Y. C  Hwang (1997). Introduction to Random Signals and Applied Kalman Filtering. Third Edition. John Wiley and Sons Inc.  pp. 250-252.
\item Grewal, M. S., A. P. Andrews (2001). Kalman Filtering Theory and Practice Using Matlab. Second Edition. John Wiley and Sons Inc.pp. 226.
\end{enumerate}

\end{document}
