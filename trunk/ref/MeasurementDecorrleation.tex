\documentclass[letterpaper,11pt]{article}

\setlength{\parindent}{0.0cm}
\setlength{\parskip}{0.2cm}

%opening
\title{Diagonalization of the Measurement Variance-Covariance Matrix}
\author{Glenn D. MacGougan}

\begin{document}

\maketitle

\begin{abstract}
In Kalman filtering, it is very useful to be able to process measurements sequentially (i.e. one at a time). This method eliminates the need for matrix inversion operations since sequential processing uses scalar measurement updates $^{[1]}$. The method requires that measurements are uncorrelated. When this is not the case, diagonalizeation of the measurement variance-covariance matrix and a transformation that is a linear combination of the original measurements into a new measurement set is performed.
\end{abstract}

\section{Methods}

\subsection{Best References}

The book by Brown and Hwang $^{[1]}$ provides excellent reference information for these method (refer to pages 250-252 for the method description, problem 6.2 on page 283 introduces measurement variance-covariance diagonalization, and pages 367-370 discuss U-D factorization).

The book by Grewel and Andrews $^{[2]}$ discusses decorrelated measurement noise (pages 221-225). Table 6.6 of this book provides a good summarized procedure for measurement decorrelation. 

\subsection{Diagonalization Example}
Problem 6.2 $^{[1]}$:

\begin{quote} 
``Consider the measurment to be a 2-tuple $[z_1, z_2]^T$, and assume that the measurment errors are correlated such that the \textbf{R} matrix is of the form

\begin{equation}
\textbf{R} = \left[ 
\begin{array}{cc}
r_{11} & r_{12} \\
r_{21} & r_{22} 
\end{array}
\right] 
\end{equation}

(a) Form a new measurment pair, $z_1'$ and $z_2'$, as a linear comination of the orginal pair such that the errors in the new pair are uncorrelated. (Hint: First, let $z_1'= z_1$ and then assume $z_2' = c_1 z_1 + c_2 z_2$ and choose the constants $c_1$ and $c_2$ such that the new measurement errors are uncorrelated.)''$^{[1pp. 283]}$
\end{quote}

Answer:

(a) Alternatively let $z_2' = z_2$ and assume $z_1' = c_1 z_1 + c_2 z_2$:

\begin{equation}
z'= \left[ 
\begin{array}{c}
c_1 z_1 + c_2 z_2  \\
z_2 
\end{array}
\right]
\end{equation}

Using variance-covariance propagation law:

\begin{equation}
\begin{array}{c}
y=f(z) \\
C_y = \frac{\delta f}{\delta z} C_z \frac{\delta f}{\delta z}^T
\end{array}
\end{equation}

Thus $\frac{\delta f}{\delta z}$ is:

\begin{equation}
\frac{\delta f}{\delta z} = \left[ 
\begin{array}{cc}
c_1 & c_2 \\
0 & 1 \\
\end{array}
\right]
\end{equation}

Thus $R'= \frac{\delta f}{\delta z} R \frac{\delta f}{\delta z}^T$ is:

\begin{equation}
\begin{array}{c}
R' = 
\left[ 
\begin{array}{cc}
c_1 & c_2 \\
0 & 1 \\
\end{array}
\right]
\left[ 
\begin{array}{cc}
r_{11} & r_{12} \\
r_{21} & r_{22}
\end{array}
\right]
\left[ 
\begin{array}{cc}
c_1 & 0\\
c_2 & 1
\end{array}
\right] \\
= 
\left[ 
\begin{array}{cc}
c_1 r_{11} + c_2 r_{12} & c_1 r_{12} + c_2 r_{22} \\
r_{21} & r_{22}
\end{array}
\right]
\left[
\begin{array}{cc}
c_1 & 0\\
c_2 & 1
\end{array}
\right]
\end{array}
\end{equation}

Results in: 
\begin{equation}
R' = 
\left[ 
\begin{array}{cc}
c_1^2 r_{11} + 2 c_1 c_2 r_{12} + c_2^2 r_{22}  & c_1 r_{11} + c_2 r_{22} \\
c_1 r_{11} + c_2 r_{22}                         & r_{22}
\end{array}
\right]
\end{equation}

Choose $c_1 = 1$ and thus $c2 = -r_{11} / r_{22}$ to make the 'new' measurements uncorrelated. This result in:

\begin{equation}
R' = 
\left[ 
\begin{array}{cc}
r_{11} - 2 \frac{r_{11}}{r_{22}}r_{12} + \frac{r_{11}^2}{r_{22}} & 0\\
0       &  r_{22}
\end{array}
\right]
\end{equation}

\subsection{Measurement Decorrelation Procedure}

\begin{quote}
''The vector-valued measurement $z = H x + v$, with correlated components of the measurement error $E[v V^T] = R$, is transformed to the measurement $z' = H' x + v'$ with uncorrelated components of the measurment error, $v' (E[v' v'^T] = D$, a diagonal matrix), by overwriting $H$ with $H' = U^{-1} H$ and $z with z' = U^{-1}z$, after decomposing $R$ to $U D U^T$, overwriting the diagonal of $R$ with $D$.``$^{[2. pp. 224]}$
\end{quote}


\section{References}
\begin{enumerate}
\item Brown, R. G., P. Y. C  Hwang (1997). Introduction to Random Signals and Applied Kalman Filtering. Third Edition. John Wiley and Sons Inc.  pp. 250-252, 283, 367-370. 
\item Grewal, M. S., A. P. Andrews (2001). Kalman Filtering Theory and Practice Using Matlab. Second Edition. John Wiley and Sons Inc.pp. 221-225.
\end{enumerate}

\end{document}
